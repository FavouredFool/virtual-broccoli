% !TEX root = Konzept.tex
\chapter{Einleitung}
Das Ziel des Projekts \dq Spielname \dq besteht darin, eine interaktive Software zu entwickeln, welche auf VR-Plattformen ausgelegt ist. In dieser sollen sogenannte \dq Haptic Gloves \dq verwendet werden, um verschiedene haptische Interaktionsmöglichkeiten anhand von unterschiedlichen spielerisch aufgezogenen Situationen in Form von Minispielen zu erforschen und deren Auswirkung auf die Erfahrung eines Nutzers zu verschriftlichen.\\
\section{Technischer Hintergrund}
Die technische Umsetzung des Projekts erfolgt in der Game Engine Unity. Dies ist eine der derzeit populärsten und vermutlich auch leistungsfähigsten Game Engines auf dem Markt. Sie bietet eine Vielzahl an Herangehensweisen zur Entwicklung von Spielen und ist durch die zugängliche Anbindung an die virtuelle Realität für das geplante Ziel gut geeignet. Zusätzlich ist Unity als Freeware für jeden Studenten kostenlos nutzbar.
\subsection{Git}
Um die gemeinsame Arbeit am Projekt besser zu organisieren, verwenden wir das Versionskontrollsystem Git, da wir den Umgang damit bereits gewohnt sind und es das mit Abstand am weitesten verbreitete moderne Versionskontrollsystem der Welt ist.
\subsection{Genutzte Hardware}
HTC VIVE Focus 3 + Haptic Gloves...
\subsection{Miro}
Damit wir Ideen zu spezifischeren Spielideen direkt visualisieren und gemeinsam besprechen können, nutzen wir im Projekt Miro. Mithilfe dieser Open-Source Software ist die gleichzeitige Bearbeitung von Grafiken wie Mindmaps oder Tabellen auf einem digitalen Whiteboard möglich.