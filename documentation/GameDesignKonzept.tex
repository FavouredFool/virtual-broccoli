% !TEX root = Konzept.tex
\chapter{Game Design}
Da das technische Fundament bereits festgelegt wurde, machten wir uns zunächst genauere Gedanken zur Spielidee. Hierfür erstellten wir ein digitales Whiteboard bei Miro, um über einen einfachen direkten Weg Ideen zu sammeln, teilen und festzuhalten. Um hierbei die Wahl der Konzepte nicht bereits im Vorhinein zu stark einzuschränken, wurde eine sehr offene und erweiterbare Spielform, in Form eines \dq Escape Rooms\dq, als Grundlage gewählt. Diese Idee wurde im Laufe der Entwicklung spezifiziert und stetig verändert, jedoch blieb erhalten, dass - ähnlich wie in einem Escape Room - ein Spieler Puzzle und Aufgaben verschiedener Arten lösen muss um das Ende zu erreichen. Dies bot die Freiheit mit vielen verschiedenen Ideen zu arbeiten, welche verschiedene Grundkonzepte der Interaktion ansprachen. Das finale Konzept der Spielidee lässt sich wie folgt beschreiben..\\
\subsection{Spielidee}
In einer magischen Welt, die zeitlich der Epoche des Mittelalters zuzuschreiben ist, sind auf mysteriöse Weise alle Farben verschwunden, da die Göttin der Farben ihrer Kräfte beraubt und in einem Turm eingesperrt worden ist. Indem ihr heiliges Diadem, welches die magischen Farbkristalle enthielt, zerbrochen wurde, hat nicht nur die Göttin ihre Kraft verloren, sondern alle Farben sind abhanden gekommen. Frei nach dem Motto \dq Farben sind das Leben Lied \dq, existiert kein Leben ohne die entsprechenden Farben, weshalb ohne die Kristalle nichts korrekt funktioniert. Da jedoch ein Teil der Kraft der Göttin immer noch in den Kristallen gespeichert ist, geben diese genügend Farbe ab, um einen kleinen Bereich in gleichnamiger Farbe etwas Leben einzuhauchen. Sollte der Spieler die einzelnen Kristalle finden und ihre Kraft über eine Maschine wieder in ihre Umgebung einfügen können, so besteht doch noch etwas Hoffnung, um die Göttin aus dem Turm zu befreien.\\
\subsection{Setting/ Look \& Feel}
Der Look des Spiels orientiert sich, wie die Spielidee bereits vermuten lässt, optisch an einer magischen, rätselhaften Fantasy Welt, die durch viele bewegbare Treppen von den Treppenhäusern aus Harry Potter inspiriert wurde. Da die Farbkristalle im gesamten Turm verteilt sind, existieren neben der Farbe Weiß, nur noch die primären Farben des subtraktivem Farbschemas Cyan, Magenta, Gelb und Schwarz. Diese Farben sind, gemäß dem Hintergrund des Spiels, bereits zu Spielbeginn punktuell in der Welt zu finden. Das führt dazu, dass das Innere des Turms in separierte Bereiche unterteilt ist, wodurch die Orientierung dem Spieler deutlich einfach fallen sollte. Jedem dieser Bereiche wurden eigene Themen zugewiesen, welche unserer Meinung nach am ehesten zur Farbe des Raums passen und eine flexible Auswahl an Rätselarten zulassen würden. Daraus ergaben sich folgende Räume, welche in einem späteren Kapitel nochmals genauer erklärt werden.
\subsubsection{Lobby}
\subsubsection{Cyan-Raum}
\subsubsection{Magenta-Raum}
\subsubsection{Dunkelraum}

\newpage
\noindent
\subsection{Spielbeginn}
Zu Beginn des Spiels befindet sich der Spieler bereits im Treppenhaus, welches den Hauptbereich des Spiels darstellt. Hier wird dem Spieler ein erster Eindruck vom Farbschema und der vertikalen Größe des Spiels gegeben. Sollte sich der Spieler etwas nach vorne bewegen, bekommt er zudem den ersten Kristall zu Gesicht, welchen er in gleicher Richtung auch direkt im ersten interaktiven Minispiel benutzen kann. Dazu aber in einem späteren Kapitel mehr.\\
\subsection{Spielfortschritt/Ziel}
Um im Spiel Fortschritt zu erzielen, muss ein Spieler die Rätsel der einzelnen Farbräume lösen und die daraus gewonnenen Farbkristalle im oben erwähnten Minispiel des Hauptraums, welcher im Folgendem als Lobby bezeichnet wird, in Farbe zu konvertieren, um so die verschiedenen Farben in die Spielwelt zurückzubringen.
Dabei ist zu beachten, dass die Räume auch nur dann erreichbar sind, sollte der Spieler die vorher abzuschließenden Rätsel abgeschlossen haben. An dieser Stelle wird eine besondere Spielmechanik relevant.
\subsubsection{Spielmechanik: Treppen}
Die Treppen der Lobby haben verschiedene Farben und nur wenn diese in die Welt zurückgebracht wurden, kann eine Treppe bewegt werden, um den Weg nach oben zu ermöglichen. Darüber hinaus zeigt sich in diesem Rätsel eine simplere Form des Interaktionskonzepts. Die Aktion des Ziehens eines farbigen Hebels führt direkt zu der Reaktion einer Auswahl gleichfarbiger Treppen, welche im Treppenhaus zu neuen Positionen rotieren. Hier zeigt sich die Stärke des Interaktionskonzepts in einer VR-Umgebung besonders, da ein Spieler nach dem benutzen eines Hebels direkt um ihn herum die Auswirkungen dieser Aktion spürt. Weil bestimmte Kristalle für bestimmte Treppen zuvor eingesetzt worden sein müssen, bekommt das Spiel einen deutlich weniger linearen Spielverlauf, was uns besonders wichtig gewesen ist. Zusätzlich greift diese Spielmechanik das Interaktionskonzept mit der Umgebung zu interagieren auf, da sich die Welt direkt nach Betätigen des Hebels verändert.\\
Weiterhin wird ein Spieler im Verlauf des Spiels verschiedene Rätsel lösen und damit mehr Farben in eine zuvor fast ausschließlich weiße Welt einführen. Dieses Feedback bietet dem Spieler eine grobe Darstellung des Spielfortschritts, führt aber ebenso zu einem Gefühl von höherem Ziel und Zweck, ein Teil dieser Welt zu sein. Dies wird weiter dadurch verstärkt, dass Rätsel des Spiels nur mit den richtigen Farben zu lösen sind und der Zweck somit nicht nur subtil in der Umgebung dargestellt wird, sondern spürbar in den darauf folgenden Rätseln ist.\\

\section{Umsetzung}
Wie am Anfang dieses Kapitels erwähnt, sollte GAME eine offene Spielform innehalten, da in allen Formen von Entwicklungen bestimmte Prozesse und Umsetzungen mehr Zeit in Anspruch nehmen können, als man zuvor gedacht hat. Der Grundgedanke des Spiels wurde hierbei jedoch nicht verändert. In verschiedenen Rätseln innerhalb der Lobby und Farbräume müssen

\newpage
\noindent
\section{Interaktionskonzepte im Spiel}
\subsubsection{Visuell}
%TURM
Um das Spiel zu beenden muss der Spieler den Turm erklimmen und somit hoch hinaufsteigen. Sobald man nahe dem höchsten Punkt der Spielwelt ist, müssen Spieler auf eine dünne Planke steigen. Hierbei hat man ein freies Sichtfeld auf die Tiefe unter sich, wodurch der Effekt dieser visuellen Interaktionsform zum Vorschein kommt. Die Reaktion auf diese Höhe kann sehr individuell sein, jedoch erzeugt eine solche Höhe starke Emotionen.\\
%DARKROOM
Wie zuvor erwähnt, ist die Einschränkung der Sicht auch ein gutes Mittel, um besondere Erfahrungen mit visuellen Interaktionen zu sammeln. Der dritte Farbraum legt dabei die Verantwortung der Sicht auf den Spieler. Im gelben Farbraum, im Folgendem auch Dunkelraum, muss der Spieler Lichtquellen in den Raum werfen, um die allumfassende Dunkelheit des Raumes punktuell zu durchbrechen. Dies soll gemacht werden, um mithilfe verschiedenster Tipps und Gegenstände, die im Raum zu finden sind, die korrekte Lösung für ein Bilderrätsel in Erfahrung zu bringen und mithilfe interaktiver Hebel den korrekten Code zu erstellen. So etwas fundamentales wie die Lichtverhältnisse eines Raumes in die Hand eines Spielers zu legen, führt zu einer deutlichen Erhöhung der Konzentration und Immersion. Dabei fiel auf, dass Spieler, die damit beschäftigt waren ihren Sichtsinn aufrechtzuerhalten, deutlich intensiver in das Spiel vertieft waren.\\
// TODO: Sicherstellen, dass die obige Aussage korrekt ist\\
%HAPTIK wegen Controller am Biespiel Rätseltisch)
Ein weiterer Punkt der visuellen Interaktion fiel eher zufällig auf als ein geplantes Feature des Spiels zu sein. Durch die SenseGloves werden die Hände in der echten Welt exakt auf die virtuelle Realität abgebildet. Wegen der in Abschnitt \dq Technisches Konzept\dq beschriebenen technischen Schwierigkeiten, muss ein Spieler gleichzeitig einen SenseGloves-Handschuh und einen VR-Controller in jeweils einer Hand halten. Die Unterschiede in der Darstellung der Hände durch diese zwei verschiedenen Geräte wird sehr schnell deutlich. Die Hand-Augen-Koordination, die die SenseGloves einem Spieler bieten, führen zu deutlicher stärkerer Immersion als die grobe Annäherung, welche ein einfacher VR-Controller bietet. Besonders bei Aufgaben die Fingerfertigkeit benötigten, wie der bereits beschriebene Rätseltisch, war neben dem haptischen Feedback das visuelle Feedback der virtuellen Hände ein wichtiges Merkmal für verstärkte Immersion.
\subsubsection{Haptik}
Haptische Interaktion wurde in diesem Projekt besonders durch die \dq SenseGloves\dq realisiert. Diese Handschuhe sind in der Lage die Bewegung der Finger einer Hand festzusetzen. Diese Eigenschaft wurde in unserem Spiel in verschiedenen Weisen genutzt. Besonders der Rätseltisch profitierte stark von den Fähigkeiten der haptischen Handschuhe, indem...\\

TODO: HIER NOCH MEHR ZUM RÄTSELTISCH WENN ER FINALISIERT IST.\\

TODO: AUCH NOCH WAS ZUM RÖHRENPUZZLE