% !TEX root = Konzept.tex
\chapter{Einleitung}
Virtual Reality oder kurz \dq VR \dq ist bereits in aller Munde. Jegliche Darstellungsmöglichkeiten wurden dadurch in hunderten von Themengebieten revolutioniert. Die Technologie wird beispiels\-weise zu Lehrzwecken verwendet, um Vorgänge oder Aufbauten zu simulieren und natürlich auch um spielerischer und vor allem realistischer an entsprechende Gebiete heranzuführen. Es florieren bereits viele neue Ideen und Projekte, welche sich die neu entdeckte Realität zu Nutze machen oder sie erweitert bzw. verbessert, um Immersion auf neuen  Ebenen zu entwickeln und mehr menschliche Sinne genauer ansprechen zu können. Von diesem Aufschwung blieb natürlich auch die Unterhaltungsbranche nicht unbeeinflusst, weswegen sich zahlreiche neue Spielideen im Einklang mit den neuen Technologien bereits in Entwicklung befinden. Diesem Trend folgend, wurde im Rahmen des Portfoliomoduls im Studiengang Medieninformatik an der Hochschule RheinMain eine Software entwickelt, die in virtueller Realität anhand unterschiedlicher spieler\-ischer Rätsel verschiedene Interaktionsmöglichkeiten erforscht. Der Aufbau und Entwick\-lungsprozess dieser Software wird im folgenden Dokument verschriftlicht.