% !TEX root = Konzept.tex
\chapter{Einleitung}
Virtual Reality oder kurz \dq VR \dq ist bereits in aller Munde. Jegliche Darstellungsmöglichkeiten wurden dadurch in hunderten von Themengebieten revolutioniert. Die Technologie wird beispielsweise zu Lehrzwecken verwendet, um bestimmte Vorgänge oder Aufbauten zu simulieren und natürlich auch um spielerischer und vor allem realistischer an entsprechende Gebiete heran zu führen. Anstatt dass es nun vorerst bei dieser bahnbrechenden Technologie belassen wird fluorieren bereits  viele neue Ideen und Projekte welche sich die neu entdeckte Realität zu nutzen machen oder sie erweitert bzw verbessert um Immersion auf ganz neuen  Ebenen zu entwickeln und noch mehr menschliche Sinne noch genauer ansprechen zu können. Von diesem Aufschwung blieb natürlich auch die Unterhaltungsbranche nicht unbeeinflusst. Weswegen sich zahlreiche neuer Spielideen im Einklang mit den neuen Technologien schon in Entwicklung befinden. Diesem Trend folgend wurde im Rahmen des Portfoliomoduls im Studiengang Medieninformatik an der Hochschule Rhein Main eine Software entwickelt, welche Interaktions-möglichkeiten bietet, die in virtueller Realität anhand unterschiedlicher spielerischer Rätsel ausgeübt werden können. Mithilfe sogenannter \dq Haptic Gloves \dq kann hierbei über Sensorik an den Fingern des Trägers haptisches Feedback zu den einzelnen Spielobjekten erzeugt werden, wodurch ein besseres Tastgefühl in der neuen Realität simuliert wird. Der Aufbau und Entwicklungsprozess dieser Software im Zusammenhang des genannten haptischen Feedbacks sowie der virtuellen Interaktion insgesamt, wird nun im folgenden Dokument verschriftlicht.\\

Old Einleitung:\\
Im Rahmen des Portfoliomoduls im Studiengang Medieninformatik an der Hochschule Rhein
Main wurde eine interaktive Software f¨ur eine VR-Umgebung entwickelt, welche Interaktionsm
öglichkeiten anhand unterschiedlicher spielerisch aufgezogener Situationen bietet. Dafür wurden
zu Teilen \dq Haptic Gloves \dq genutzt, welche haptisches Feedback an den Fingern eines Nutzers
erzeugen können. Der Entwicklungsprozess sowie die gewonnenen Erfahrungen werden im
Folgenden verschriftlicht.